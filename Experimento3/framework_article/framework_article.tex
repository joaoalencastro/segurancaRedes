% Autor: João Fiuza de Alencastro
% Disciplina: Segurança de Redes
% Relatório 3
\documentclass[journal]{IEEEtran}
\usepackage{listings}
\usepackage[utf8]{inputenc}
\usepackage{graphicx}




\begin{document}

\title{Relatório Frameworks}


\author{João~Fiuza~de~Alencastro~15/0131933}% <-this % stops a space




% make the title area
\maketitle


\begin{abstract}
Relatório destinado à matéria de Segurança de Redes do Departamento de engenharia Elétrica da Universidade de Brasília. Experimento realizado enfatizando a utilização de frameworks e aplicações prontas.
\end{abstract}

\begin{IEEEkeywords}
Segurança, redes, Framework, NMap, ports, TCP, UDP, vulnerabilidade.
\end{IEEEkeywords}


\IEEEpeerreviewmaketitle



\section{Introduction}
\IEEEPARstart{R}{ealizar} experimentos utilizando ferramentas específicas, melhor desenvolvidas e de fácil acesso. Frameworks oferecem a vantagem de juntar mais de uma ferramenta de descoberta de vulnerabilidades e de testes de penetração em uma só interface. Testes de penetração são realizados de forma correta quando seguem uma sequência de ações, construindo uma base de conhecimento no início e evoluindo à medida que são realizados testes, por este motivo aplicações específicas permitem ataques mais bem estruturados.


\section{Conclusion}



\begin{thebibliography}{1}

\bibitem{}


\end{thebibliography}



\end{document}


