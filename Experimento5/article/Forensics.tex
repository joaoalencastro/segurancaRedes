% Autor: João Fiuza de Alencastro
% Disciplina: Segurança de Redes
% Relatório 5
\documentclass[journal]{IEEEtran}
\usepackage{listings}
\usepackage[utf8]{inputenc}
\usepackage{graphicx}
\usepackage[colorlinks=true,urlcolor=red,citecolor=blue,linkcolor=blue]{hyperref}




\begin{document}

\title{Relatório Forense}


\author{João~Fiuza~de~Alencastro~15/0131933}% <-this % stops a space




% make the title area
\maketitle


\begin{abstract}
Relatório destinado à matéria de Segurança de Redes do Departamento de engenharia Elétrica da Universidade de Brasília. Experimento realizado a fim de explorar a vasta área da ciência forense computacional. Será feita uma abordagem teórica, assim como uma abordagem prática.
\end{abstract}

\begin{IEEEkeywords}
Segurança, redes, Forensics, Autopsy, p0f, imaging, carving, digital, physical environment.
\end{IEEEkeywords}


\IEEEpeerreviewmaketitle



\section{Introduction}
\IEEEPARstart{O} termo forense, surge no direito, e é relacionado ao solucionamento de crimes. A Ciência Forense é uma união de aplicações a fim de desvendar o crime, às vezes é uma rotina de testes, outras vezes são só buscas simples. Porém, aqui será estudado um ramo específico da Forense, a Ciência Forense Computacional, voltado somente para a prática dentro do mundo computacional, mais precisamente o armazenamento digital. \par
Ao pensar em Ciência Forense, devem ser feitos dois questionamentos: "Quem?" e "Quando?". E o armazenamento digital comumente deixa rastros dessas duas perguntas. \par
Um exemplo encontrado em [2], demonstra a importância dos rastros virtuais para a justiça brasileira: "Na justiça trabalhista, a Computação Forense adquiriu extrema importância, já que a partir da portaria 1510 de 2009 do MTE, o Registro Eletrônico deve ser utilizado em todas as empresas em território nacional, descontinuando o arcaico relógio e as assinaturas em folhas de sulfite, devendo emitir um comprovante para o funcionário. Já no ano de 2012 a portaria 373/12, habilitou sistemas alternativos para a marcação do ponto, permitindo o uso de aplicativos, mensagem de texto, softwares específicos, código de barras, basicamente qualquer sistema que garanta a integridade dos registros". Todos os exemplos citados servem como evidências de auditoria e podem ser utilizados no tribunal. \par
A Forense Computacional atualmente, é dividida em três partes: Análise (Digital), Coleta (imaging) e a Extração (carving). Essa divisão mantém uma organização do fluxo de trabalho. Assim como testes de penetração, a prática forense deve ser realizada sequencialmente, de forma responsável, para que não haja perdas de evidências e provas valiosas.

\subsection{Digital Forensics (Análise)}
A análise 
%referências
%https://www.sciencedirect.com/science/article/pii/S1742287610000368 
%https://en.wikipedia.org/wiki/Digital_forensics

\subsection{Imaging Forensics (Coleta)}


\subsection{Carving Forensics (Extração)}



\section{Conclusion}




\begin{thebibliography}{1}

\bibitem{significado}
https://www.significados.com.br/forense/
\bibitem{rastros_virtuais}
https://www.contabeis.com.br/artigos/5035/seguindo-os-rastros-virtuais-computacao-forense/


\end{thebibliography}



\end{document}


